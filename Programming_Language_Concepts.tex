\documentclass{article}
\usepackage[utf8]{inputenc}
\usepackage[document]{ragged2e}

\usepackage{xcolor}
\usepackage{varwidth}

\oddsidemargin = 4pt
\textwidth = 500pt

\begin{document}

\begin{center} 
    \textsc{\large University of Southampton}       \\[1.5cm]
    \textsc{\LARGE Programming Language Concepts}   \\[0.5cm]
    \textsc{\normalsize April 2017}                 \\[3.0cm]
    
    \emph{  \large Bark of the Hound Language}      \\[0.5cm]
    \emph{  \large Authors:}                        \\
    
    \textsc{Fraser Crossman} \emph{(fc4g15)}        \\
    \textsc{Anish Katariya} \emph{(ak7n14)}         \\
\end{center}

\newpage

\tableofcontents

\newpage

% Colour definitions for program syntax highlighting

\definecolor{variable}  {rgb}{0.01, 0.28, 1.00}
\definecolor{type}      {rgb}{0.20, 0.20, 0.60}
\definecolor{operator}  {rgb}{0.00, 0.00, 0.00}
\definecolor{stringlit} {rgb}{0.80, 0.00, 0.00}
\definecolor{comment}   {rgb}{0.00, 0.50, 0.00}
\definecolor{token}     {rgb}{0.00, 0.20, 0.70}
\definecolor{error}     {rgb}{1.00, 0.00, 0.20}

% Commands for using predefined colours

 % trailing space is important

\newcommand{\variable}  [1]{\textcolor{variable}    {#1} } 
\newcommand{\type}      [1]{\textcolor{type}        {#1} }
\newcommand{\operator}  [1]{\textcolor{operator}    {#1} }
\newcommand{\stringlit} [1]{\textcolor{stringlit}   {#1} }
\newcommand{\comment}   [1]{\textcolor{comment}     {#1} }
\newcommand{\token}     [1]{\textcolor{token}       {#1} }
\newcommand{\error}     [1]{\textcolor{error}       {#1} }

% Commands for inserting well formatted programs

\newcommand{\program}[1]{
    %\emph{\textbf{Example:}}\\
    \noindent\texttt{#1}
}
\newcommand{\tab}[1]{\hskip 3em #1}

% -- Tokens from lexer.mll --
% Program Structure
\newcommand{\BEGIN}{\token {begin}}
\newcommand{\END}{\token {end}}
\newcommand{\INPUT}[1]{\variable {args#1}}
\newcommand{\OC}{\variable {OUTPUT\_COUNT}}
\newcommand{\ES}{\variable {\_\_empty\_string}}

% Types
\newcommand{\INT}[1]{\type{int} #1}
\newcommand{\STR}[1]{\type{str} #1}
\newcommand{\BOOL}[1]{\type{bool} #1}
\newcommand{\SET}[1]{\type{set} #1}

% If then else Statements
\newcommand{\IF}{\token {if}}
\newcommand{\ELSE}{\token {else}}
\newcommand{\THEN}{\token {then}}
\newcommand{\FI}{\token {fi}}

% For loop definition
\newcommand{\FOR}{\token {for}}
\newcommand{\IN}{\token {in}}
\newcommand{\DO}{\token {do}}
\newcommand{\ROF}{\token {rof}}

% Print for output from programs
\newcommand{\PRINT}{\token {print}}

% Assignment token
\newcommand{\LET}{\token {let}}

% Mathematical operators
\newcommand{\ASS}{\operator {$=$}}
\newcommand{\PLUS}{\operator {+}}
\newcommand{\MINUS}{\operator {-}}
\newcommand{\DIVIDE}{\operator {/}}
\newcommand{\TIMES}{\operator {*}}
\newcommand{\MOD}{\operator {\%}}

% String and Set operators
\newcommand{\INSERT}{\operator {Insert}}
\newcommand{\SETMINUS}{\operator {SetMinus}}
\newcommand{\CON}{\operator {\^{}}}

% Boolean values
\newcommand{\TRUE}{\token {true}}
\newcommand{\FALSE}{\token {false}}

% Boolean operators
\newcommand{\LT}{\operator {$<$}}
\newcommand{\GT}{\operator {$>$}}
\newcommand{\LTE}{\operator {$<=$}}
\newcommand{\OR}{\operator {$||$}}
\newcommand{\AND}{\operator {$\&\&$}}
\newcommand{\GTE}{\operator {$>=$}}
\newcommand{\EQ}{\operator {$==$}}
\newcommand{\NEQ}{\operator {$!=$}}
\newcommand{\NOT}{\operator {$!$}}

% String literal
\newcommand{\STRING}[1]{\stringlit{"#1"}}

% Comments
\newcommand{\COMMENT}[1]{\comment{/*#1*/}}

% Errors
\newcommand{\ERROR}[1]{\error{#1}}

\section{Introduction}

The Bark of the Hound Language is a domain specific programming language 
capable of manipulating sets through the use of the \INSERT and \SETMINUS 
operators. Using this language were able to take finite languages as input 
and manipulate them to create new languages. The language also includes 
features found in general purpose programming languages such as printing, 
looping, conditionals, mathematical operators, and boolean operators that can 
be used to solve a wide variety of problems. All inputs are assumed to be a 
sequence of finite sets followed by a positive integer, which represents the 
maximum size of the output set.\\

\program{
\{a, bc, de\}\\
3
}

\section{Syntax}
\subsection{Basics}
A well structured program written in the language starts with a \BEGIN token 
and ends with an \END token. Comments in the language are written between two 
backslash star sequences (\COMMENT{ comment }) 

\subsection{Variable Types}
The language has support for three primitive types; Integer, String and Boolean 
in addition to the Set type. Each variable declaration is preceded by the 
keyword \LET and followed by a semicolon.

\program{
    \tab    \LET \INT{\noindent w};
    \tab    \COMMENT{ Integer named w }
    \tab    \LET \STR{ x};
    \tab    \COMMENT{ String named x }\\
    \tab    \LET \BOOL{y};
    \tab    \COMMENT{ Boolean named y }
    \tab    \LET \SET{ z};
    \tab    \COMMENT{ Set named z }\\
}

\subsection{Operations}
\subsubsection{Integers}
The language has support for the following mathematical operators: addition 
( \PLUS), subtraction ( \MINUS), division ( \DIVIDE), multiplication ( \TIMES), 
and modulus ( \MOD). All mathematical operators are left associative. Brackets 
can be used to override their precedence.\\

\program{
    \tab    \LET \INT{x} \ASS 4;
    \tab    \INT{x} \ASS \INT{x} \PLUS 1 \TIMES 5;
    \tab    \COMMENT{ x = 9 }\\
}

\subsubsection{Strings}
For string manipulation the language has support for string concatenation 
( \CON) operator.\\

\program{
    \tab    \LET \STR{x} \ASS \STRING{Julian};
    \tab    \LET \STR{y} \ASS \STRING{Rathke};\\
    \tab    \LET \STR{ z} \ASS \STR{x} \CON \STRING{ } \CON \STR{y};
    \tab    \COMMENT{ z = "Julian Rathke" }\\
}

\subsubsection{Booleans}
The language has support for boolean operators such as equals ( \EQ), not equal 
( \NEQ), less than ( \LT), greater than ( \GT), less than or equal to ( \LTE), 
greater than or equal to ( \GTE) , and ( \AND) , or ( \OR), and not ( \NOT)\\

\program{
    \tab    \LET \BOOL{a} \ASS (3 \GT 4);
    \tab    \COMMENT{ a = false}\\
}

\subsubsection{Sets}
The language has support for both set minus ( \SETMINUS) and set insert 
( \INSERT) operations. Sets also maintain alphabetical order when stored and 
printed to standard output.

\program{
    \tab    \LET \SET{x};\\
    \tab    \SET{x} \ASS \SET{x} \INSERT \STRING{A};
    \tab    \COMMENT{ x = \{A\} }\\
    \tab    \SET{x} \ASS \SET{x} \INSERT \STRING{C};
    \tab    \COMMENT{ x = \{A, C\} }\\
    \tab    \SET{x} \ASS \SET{x} \INSERT \STRING{B};
    \tab    \COMMENT{ x = \{A, B, C\} }\\
}

\subsubsection{Print}
The language provides the ability to print \textbf{any} data type to standard 
output using the \PRINT statement.\\

\section{Control Flow}
The language has support for conditional if else statements and for loops which 
may be nested.

\subsection{If- Then - Else}
If then else statements use the following syntax:\\
\IF {$<$}condition{$>$} \THEN {$<$}statement{$>$} \FI\\
or\\
\IF {$<$}condition{$>$} \THEN {$<$}statement{$>$} \ELSE {$<$}statement{$>$} \FI\\

\program{
    \BEGIN\\
    \tab    \IF (3 \LT 4) \THEN\\
    \tab    \tab    \PRINT "GREEN";\\
    \tab    \ELSE\\
    \tab    \tab    \PRINT "RED";\\
    \tab    \FI
    \tab    \COMMENT{ "GREEN" is printed to standard output }\\
    \END
}

\subsection{For Loop}
The language has support for regular for loop and enhanced for loops. The for 
loops can be described written as either:\\
\FOR {$<$}condition{$>$} \DO {$<$}statement{$>$} \ROF\\
or\\
\FOR {$<$}string{$>$} \IN {$<$}set{$>$} \DO {$<$}statement{$>$} \ROF\\

\program{
    \BEGIN\\
    \tab    \LET \SET{s1us2};\\
    \tab    \LET \STR{temp};\\
	\tab    \FOR \STR{inp1} \IN \INPUT{0} \DO\\
	\tab    \tab    \STR{temp} \ASS \STRING{a} \CON \STR{inp1};\\
	\tab    \tab    \SET{s1us2} \ASS \SET{s1us2} \INSERT \STR{temp};\\
	\tab    \ROF\\
    \END
}

\section{Additional features}
\subsection{Programmer Convenience}
This language provides many programmer conveniences to aid the writing of 
simple programs. As previously noted, all sets store their strings 
alphabetically which is advantageous when printing only a subset of a language 
stored in a set variable. Variables are also initialised on declaration to 
prevent the need to initialise a variable which will later be reassigned. 

\begin{table}[h!]
    \centering
    \begin{tabular}{|c|c|}
        \hline
        \emph{Type}     & \emph{Initialised Value}\\\hline
        \emph{Integer}  & 0\\                       \hline
        \emph{String}   & ""\\                      \hline
        \emph{Boolean}  & false\\                   \hline
        \emph{Set}      & \{:\}\\                   \hline
    \end{tabular}
    \caption{Depicts data types and automatically initialised values}
    \label{table:1}
\end{table}

In order to ensure that the output of a program does not exceed the value 
specified by the trailing integer in input the binding \OC is added so that it 
can be referenced and used in programs. This is especially useful when working 
with infinite sets.

\program{
    \BEGIN\\
    \tab    \LET \SET{aStar};
    \tab    \LET \STR{aString};
    \tab    \LET \INT{count};\\
	\tab    \FOR \INT{count} \LT \INT{\OC} \DO\\
	\tab    \tab    \INT{count} \ASS \INT{count} \PLUS 1;\\
	\tab    \tab    \SET{aStar} \ASS \SET{aStar} \INSERT \STR{aString};\\
	\tab    \tab    \STR{aString} \ASS \STR{aString} \CON \STRING{a};\\
	\tab    \ROF\\
	\tab    \PRINT \SET{aStar};
	\tab    \COMMENT{ a* is printed - up to the length specified }\\
    \END
}

For the purposes of readability, it is also possible to use a predefined 
constant to represent the empty sting named \ES.

\subsection{Type Checking}
The symbols \PLUS, \MINUS, \DIVIDE, and \TIMES must be used with integer values 
only. Similarly, the string concatenation operator \CON must be used with only 
strings. The boolean operators \GT, \GTE, \LT, and \LTE must also be used with 
integer values only however, \OR, \AND, and \NOT must only be used with Boolean 
type expressions.

The \EQ and \NEQ operators may be used by integer, string, or Boolean values. 
Finally the operators \INSERT and \SETMINUS must only be used with a set as the 
first argument and a string as the second. This is because, in this language, 
sets only contain strings. It is convenient for the programmer that these rules 
are enforced by the language interpreter as it means that unintentional 
problems with a program can be found early in development.

\newpage

\subsection{Error Messages}
The error messages displayed are informative and indicate to the programmer
the type of error that has occurred. This proves to be very useful when 
debugging user written programs.

\subsubsection{Syntax Errors}
If any of the aforementioned type checks fail or a general syntax error is 
found then the following error is displayed: 
\texttt{\ERROR{Fatal error: exception Parsing.Parse\_error}}

\subsubsection{Invalid Variable References}
If a variable is not present in the current environment, an error specific to 
the specified type is displayed. The general form of this error is:

\texttt{
    \ERROR{
        Fatal error: exception Failure("Variable [type] [var name] Not Declared 
        as a [type name] type. Hint: Maybe try - let [type] [var name] = 
        [default value];")
    }
}

An example of this type of error where x is an integer referenced, but not 
present, in the current environment is shown below:

\program{
    \ERROR{
        Fatal error: exception Failure("Variable int x Not Declared as an 
        integer type. Hint: Maybe try - let int x = 0;")
    }
}

If the token denoting the type of a variable is missing the following error is 
displayed: \texttt{\ERROR{Fatal error: exception Failure("lexing: empty token")}}

If a variable is reused in the subject of a for loop then an error is thrown.
\texttt{\ERROR{Variable [var name] is already in use. Try changing variable name.}}

Our language allows reference to arguments. If an undefined argument is 
specified then the following error is displayed: \texttt{\ERROR{Invalid Input 
[arg ref].Enter valid args\#.}}

\subsubsection{Division by Zero Error}
If a division by 0 is performed this error is displayed: 
\texttt{\ERROR{Fatal division by 0 error.}}

\subsubsection{Print Error}
If a set, for whatever reason, fails to print correctly when using the print 
keyword then the following error is displayed: 
\texttt{\ERROR{Set undefined. Ensure the set is correctly defined.}}
\end{document}